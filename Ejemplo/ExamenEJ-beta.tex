% \documentclass[10pt,a4]{aleph-examen-beta}
\documentclass[10pt,a4,respuestas]{aleph-examen-beta}

% -- Paquetes 
\usepackage{aleph-comandos}


% -- Datos del libro
\institucion{Escuela de Ciencias Físicas y Matemática}
\carrera{Enfermería}
\asignatura{Bioestadística}
\autor{Mat. Andrés Merino}
\tema{Examen No. 1: Funciones}
\fecha{Semestre 2022-1}


\fuente{montserrat}
\logouno[0.1\textwidth]{Logo_EPN}
% \logodos[1.0cm]{Logos/logo01}


% \hoja{Hoja de enunciados}
% \materia{Cálculo Vectorial}
% \examen{Examen del Segundo Bimestre}
% %\tema{Integrales}
% \autor{Departamento de Formación Básica}
% \fecha{Miércoles 8 de agosto de 2018 (120 minutos)}
% \logouno[1.5cm]{Logo_EPN.eps}
% \logodos[2.5cm]{Logo_DFB.eps}



\begin{document}

\encabezado

% \datosestudiante
\section*{Preguntas}

\begin{preguntas}
\item
    Encontrar los extremos de la función
    \[
		\funcion{f}{S}{\R}{(x,y,z)}{\frac{x^2}{2} + \frac{y^2}{2} + \frac{z^2}{2}.}
	\]

\begin{respuesta}
    La función no tiene extremo.
\end{respuesta}

\item
    Escriba la definición de derivada de una función en un número.
	
\begin{respuesta}
    Dada una función $\func{f}{I}{\R}$ y $a\in I$, se dice que $f$ es derivable en $a$ si existe
    \[
        \lim_{h\to 0}\frac{f(x+h)-f(x)}{h}
    \]
\end{respuesta}

\end{preguntas}


\end{document}
